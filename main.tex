\documentclass[12pt, letterpaper]{article}

% Page Size and Margins
\setlength{\topmargin}{-0.5in}
\setlength{\oddsidemargin}{0in}
\setlength{\evensidemargin}{0in}
\setlength{\textwidth}{6.5in}
\setlength{\textheight}{9in}
\setlength{\headheight}{15pt}

% Font
\usepackage{times} 
\usepackage[style=apa,backend=biber]{biblatex}
\addbibresource{references.bib} % your .bib file name

% Footer
\pagestyle{plain}

% Line Spacing
\renewcommand{\baselinestretch}{2.0}\normalsize

% Redefine Heading Sizes to Match Regular Text
\makeatletter
\renewcommand\section{\@startsection {section}{1}{\z@}%
                                   {-3.5ex \@plus -1ex \@minus -.2ex}%
                                   {2.3ex \@plus.2ex}%
                                   {\normalfont\normalsize\bfseries}}
\renewcommand\subsection{\@startsection {subsection}{2}{\z@}%
                                      {-3.25ex \@plus -1ex \@minus -.2ex}%
                                      {1.5ex \@plus .2ex}%
                                      {\normalfont\normalsize\bfseries}}
\makeatother

% Title
\title{\textbf{MISQ LaTeX Template [Your Article Title]}}
\date{} % No date
\begin{document}

\maketitle

% Abstract
\begin{abstract}
\renewcommand{\baselinestretch}{1.5}\normalsize
Your abstract goes here.
The first page of your manuscript should include the following: (1) title, (2) abstract, and (3) 5-10 keywords related to the paper’s topic. Your introduction should begin on page 2 of the manuscript. 
\end{abstract}

% Keywords
\renewcommand{\baselinestretch}{1.0}\normalsize
\textbf{Keywords:} Your keywords go here.

% Introduction
\newpage
\section{\centering{INTRODUCTION}}
\renewcommand{\baselinestretch}{2.0}\normalsize
Your introduction goes here.
This document is meant to serve as a visualization of MIS Quarterly’s formatting guidelines. The layout, text, and headings are formatted in accordance with the guidelines detailed on our website. 
Use this document as a template to copy and paste your manuscript before submitting it to MISQ.

Test References: For example, we construct a large relational database \parencite{codd1970relational} to collect a large
amount of data. In order to communicate the structure of this database, we
use an Entity-Relationship diagram \parencite{chen1976entity}. We then analyze
this data and learn great things about how to improve the world using information systems. This is for \parencite{author1year}, And this is \parencite{author2year}, this is for \parencite{brown2023fault}, \parencite{gupta2018economic}

\section{\centering \MakeUppercase {Major Header}}
Your content for major header 2 goes here.

\subsection{\centering {First Subhead}}
Your content for subheading 1 goes here.

\subsubsection{Second Subhead}
Your content for subheading 2 goes here.

% Methodology, Results, Discussion, and Conclusion sections can be added here

% References
\newpage
\section*{\centering{REFERENCES}}
\printbibliography[heading=none]

% Appendix
\newpage
\section*{\centering{APPENDIX}}
Your appendices go here.

\end{document}
